\documentclass[11pt,a4paper,addpoint,answers]{exam}
\usepackage[T1]{fontenc}
\usepackage[utf8]{inputenc}
\usepackage[]{lmodern}
\usepackage[french]{babel}
%\usepackage{mathtools}
\usepackage[margin=2cm]{geometry} %layout
\usepackage{graphicx}
\usepackage{booktabs} % for much better looking tables
% Put the bibliography in the ToC
\usepackage[nottoc,notlof,notlot]{tocbibind}
% Alter the style of the Table of Contents
\usepackage[titles]{tocloft}

\usepackage[pdfauthor={Axel Viala},
  pdftitle={Examen ESGI OS-03},
  pagebackref=true,%
  colorlinks=true,%
  linkcolor=green,%
  %urlcolor=green!70!black,
  pdftex]{hyperref}
\usepackage[ampersand]{easylist}
\renewcommand{\solutiontitle}{}

\author{Axel Viala <axel.viala@darnuria.eu>}
\title{Auto evaluation connaissances 04\\Bases: mémoire - durée 15 minutes}
\date{28 novembre 2019}

\begin{document}
  \maketitle
  \makebox[\textwidth][l]{Nom et Prénom:\hrulefill}
  \makebox[\textwidth][l]{Correcteur: Nom et Prénom:\hrulefill}
  \makebox[\textwidth][l]{Classe:\hrulefill}  
  \textbf{Objectifs:} Le but des contrôles de connaissances en début de cours est pour vous de vérifier où vous
  en êtes par rapport au cours précédent.
  \newline
  Pour moi un moyen de vérifier que la pédagogie est adaptée à la classe.
  \newline
  \textbf{Notation:} Les points sont indiqués à titre d'information la notation peut changer pour
  des raisons d'harmonisation.
  \newline
  \textbf{Entre-correction:} Vous pourrez faire la correction d'une autre copie avec tout document
  autorisé. Cette correction aura lieu au retour de la pause si vous arrivez à l'heure et
  améliorera la note initiale du correcteur.
  \begin{questions}

    \question[1] Quelles assertions générales sur malloc sont vraies? (0 ou plus choix possibles)
    \begin{checkboxes}
        \CorrectChoice malloc peut avoir recours a un syscall pour obtenir de la mémoire
        \choice En cas d'accès non authorisé à la \emph{Netsphere}\footnote{Manga et anime \emph{BLAME!} de Tsutomu Nihei} la \emph{SafeGuard} peut venir vous terminer
        \CorrectChoice La mémoire peut être donnée a la demande (l'espace est donnée mais pas encore la mémoire réelle)
        \CorrectChoice La mémoire allouée est dans le segment de tas (HEAP)
        \choice La mémoire allouée est dans le segment de pile (STACK)
    \end{checkboxes}
  
    \question[1] Un espace d'adressage sur $32 bits$ peut aller jusqu'à:
    \begin{checkboxes}
      \choice $2^{32}$ soit \texttt{0x100000000}
      \CorrectChoice $2^{32} - 1$ soit \texttt{0xFFFFFFFF} \emph{Erratum: dans le sujet il y avait une coquille c'est bien $2^{32 - 1}$ et non $2^{31}$}
      \choice $2^{16} - 1$ soit \texttt{0xFFFF}
      \choice \texttt{0xDEADBEEF}
    \end{checkboxes}
    
    \question[1] À propos de la mémoire sur un ordinateur et OS (moderne) comme vos PC portables
    \begin{checkboxes}
      \choice En C sur Linux dans un processus on manipule des adresses physiques
      \CorrectChoice Elle est paginée (découpée en pages) et segmentée
      \CorrectChoice Chaque procesuss à sa mémoire virtuelle
      \choice La mémoire permet de se connecter à la \emph{Netsphere} 
      \CorrectChoice Un composant matériel dédié est requis pour traduire les adresses de virtuel à physique
      \choice En C sur Linux si j'adresse \emph{en dehors} de mon espace d'adressage authorisé rien ne se passe
    \end{checkboxes}

    \question[2] Dessiner (au dos) un schema de l'espace d'adressage virtuel d'un processus.
    Indiquer au moins 3 segments indispensables.
    \ifprintanswers{\emph{Réponses:} Voir cours ;)} \fi

  \end{questions}
\end{document}

\documentclass[11pt,a4paper,addpoint,answers]{exam}
\usepackage[T1]{fontenc}
\usepackage[utf8]{inputenc}
\usepackage[]{lmodern}
\usepackage[french]{babel}
%\usepackage{mathtools}
\usepackage[margin=2cm]{geometry} %layout
\usepackage{graphicx}
\usepackage{booktabs} % for much better looking tables
% Put the bibliography in the ToC
\usepackage[nottoc,notlof,notlot]{tocbibind}
% Alter the style of the Table of Contents
\usepackage[titles]{tocloft}

\usepackage[pdfauthor={Axel Viala},
  pdftitle={Examen ESGI OS-03},
  pagebackref=true,%
  colorlinks=true,%
  linkcolor=green,%
  %urlcolor=green!70!black,
  pdftex]{hyperref}
\usepackage[ampersand]{easylist}
\renewcommand{\solutiontitle}{}

\author{Axel Viala <axel.viala@darnuria.eu>}
\title{2019-11-14: Auto evaluation connaissances 03\\Bases: Processus et mémoire - durée 15 minutes}
\date{14 novembre 2019}

\begin{document}
  \maketitle
  \makebox[\textwidth][l]{Nom et Prénom:\hrulefill}
  \makebox[\textwidth][l]{Correcteur: Nom et Prénom:\hrulefill}
  \makebox[\textwidth][l]{Classe:\hrulefill}  
  \textbf{Objectifs:} Le but des contrôles de connaissances en début de cours est pour vous de vérifier où vous
  en êtes par rapport au cours précédent.
  \newline
  Pour moi un moyen de vérifier que la pédagogie est adaptée à la classe.
  \newline
  \textbf{Notation:} Les points sont indiqués à titre d'information la notation peut changer pour
  des raisons d'harmonisation.
  \newline
  \textbf{Entre-correction:} Vous pourrez faire la correction d'une autre copie avec tout document
  autorisé. Cette correction aura lieu au retour de la pause si vous arrivez à l'heure et
  améliorera la note initiale du correcteur.
  \begin{questions}

    \question[1] Quelles assertions sur les processus sont vraies: (0 ou plus choix possibles)
    \begin{checkboxes}
        \CorrectChoice Un processus possède sa propre mémoire
        \CorrectChoice Un processus possède ses fils de calcul (threads)
        \choice Un processus peut accèder par défaut à la mémoire de tout les autres processus
        \CorrectChoice Par défaut un processus parent peut accèder à des ressources de son enfant
        \CorrectChoice Pour partager des ressources un processus doit faire une demande particulière
        \CorrectChoice Un processus peut forker et executer un autre programme
        \choice Un processus peut faire le café 
        \ifprintanswers
          \emph{En pratique oui.}
        \fi
    \end{checkboxes}
  
    \question[1] Que veux dire \texttt{pid}.
    \ifprintanswers
    \begin{solution}
        Process IDentifier
    \end{solution}
    \else
    \vspace{2in}
    \fi
    
    \question[1] À quoi sert la fonction \texttt{fork}.
    \ifprintanswers
    \begin{solution}
        Elle sert à dupliquer un processus. Suite au fork on a deux processus,
        l'enfant et le parents qui s'executent et partagent des ressources.
    \end{solution}
    \else
    \vspace{2in}
    \fi
    
    \question[2] Une segmentation fault peut être levée si: (0 choix ou plus possibles)
    \begin{checkboxes}
        \CorrectChoice Si on accède à de la mémoire non allouée
        \choice Si vous n'aimez pas RainbowDash
        \choice Si on accède à une variable locale déclarée
    \end{checkboxes}

    \question[1] Après un fork reussi combien a-t-on de processus:
    \begin{checkboxes}
        \CorrectChoice 2
        \choice 42
        \choice 3
        \choice 1
    \end{checkboxes}

    \question[1] Quels sont les principaux segments mémoires de l'espace mémoire d'un processus ?
    (0 ou plus choix possibles)
    \begin{checkboxes}
        \CorrectChoice Data: segments de données connus à la compilation
        \CorrectChoice Code: Instructions en lecture seule
        \choice Ring: L'espace unique pour les gouverner tous
        \CorrectChoice Stack: données de la pile
        \CorrectChoice Kernel: Adresses réservées au Kernel
        \CorrectChoice Heap: Zone allouée par malloc
        \CorrectChoice Memory Mapping Segment: Espaces des bibliothèques dynamiques
        \choice The good place: L'espace réservé aux bonnes valeurs
    \end{checkboxes}
  \end{questions}
\end{document}

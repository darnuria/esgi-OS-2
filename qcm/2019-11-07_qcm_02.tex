\documentclass[11pt,a4paper,addpoint,answers]{exam}
\usepackage[T1]{fontenc}
\usepackage[utf8]{inputenc}
\usepackage[]{lmodern}
\usepackage[french]{babel}
%\usepackage{mathtools}
\usepackage[margin=2cm]{geometry} %layout
\usepackage{graphicx}
\usepackage{booktabs} % for much better looking tables
% Put the bibliography in the ToC
\usepackage[nottoc,notlof,notlot]{tocbibind}
% Alter the style of the Table of Contents
\usepackage[titles]{tocloft}

\usepackage[pdfauthor={Axel Viala},
  pdftitle={Examen ESGI OS-01},
  pagebackref=true,%
  colorlinks=true,%
  linkcolor=green,%
  %urlcolor=green!70!black,
  pdftex]{hyperref}
\usepackage[ampersand]{easylist}
\renewcommand{\solutiontitle}{}

\author{Axel Viala <axel.viala@darnuria.eu>}
\title{2019-10-17: Auto evaluation connaissances 02 - Entrées sorties et généralités OS - durée 15 minutes}

\begin{document}
  \maketitle
  \makebox[\textwidth][l]{Nom et Prénom:\hrulefill}
  \makebox[\textwidth][l]{Correcteur: Nom et Prénom:\hrulefill}
  \makebox[\textwidth][l]{Classe:\hrulefill}
  \textbf{Objectifs:} Le but des contrôles de connaissances en début de cours est pour vous de vérifier où vous
  en êtes par rapport au cours précédent.
  \newline
  Pour moi un moyen de vérifier que la pédagogie est adaptée à la classe.
  \newline
  \textbf{Notation:}: Les points sont indiqués à titre d'information la notation peut changer pour
  des raisons d'harmonisation.
  \newline
  \textbf{Entre-correction:} Vous pourrez faire la correction d'une autre copie avec tout document
  autorisé. Cette correction aura lieu au retour de la pause si vous arrivez à l'heure et
  améliorera la note initiale du correcteur.
  \begin{questions}

    \question[1] La \texttt{STDIN}, \texttt{STDOUT}, \texttt{STDERR} sous les systèmes d'exploitation de la famille des UNIX sont :
    (0 ou plus choix possibles)
    \begin{checkboxes}
        \CorrectChoice Sont appelés des flux ou \textit{stream} en anglais
        \CorrectChoice Accessibles comme un fichier : on peut utiliser par exemple \texttt{read}, \texttt{write} dessus
        \CorrectChoice Par défaut \texttt{printf} écrit sur la \texttt{STDOUT}
      \end{checkboxes}

    \question[2] Expliquer de façon succincte le paradigme (choix technologique) \og{}Tout est fichier\fg{}.
    Comment il s'illustre par exemple sur Linux : (2 lignes max)
    \ifprintanswers
    \begin{solution}
        Une majorité des abstractions offertes par le kernel sont accessibles par une interface à
        base de fichier. Par exemple accèder au disque ou écrire sur la console.
    \end{solution}
    \else
    \vspace{2in}
    \fi

    \question[2] Un chemin sur un système d'exploitation Unix (MacOS ou Linux)\\
    par exemple \texttt{/home/axel/cat.jpg} sert à: (0 ou plus choix possibles)
    \begin{checkboxes}
        \CorrectChoice Accéder à une ressource (ex un fichier sur le disque) dans l'arborescence de fichiers
        \CorrectChoice Proposer une organisation des ressources (notament les fichiers)
        \choice C'est représenté comme un entier
        \CorrectChoice C'est représenté comme une sequence de caractères séparé par des \texttt{/}.
    \end{checkboxes}
    \newpage

    \question[2] Vous avez un dossier \texttt{boring\_work} qui contient un fichier \texttt{exo.c} et un dossier\\
    \texttt{hidden\_manga} contenant 2 fichiers \texttt{naussicaa.cbz}, \texttt{akira.cbz}.\\
    Combien avez vous de fichiers sur l'aborescence en dessous de \texttt{boring\_work} (\texttt{boring\_work} est inclus)?
    (on exclut les dossiers \texttt{.} et \texttt{..})\\
    Dessinez cette arboresence, on exclura ce qu'il y a au dessus de \texttt{boring\_work}.
    \ifprintanswers
    \begin{solution}
        5 fichiers, 2 dossiers, 3 fichiers sur le disque.
    \end{solution}
    \else
    \vspace{3in}
    \fi

    \question[4] Quelles propositions sont vraies à propos d'un kernel (noyau) de système d'exploitation: (0 ou plus choix possibles)
    \begin{checkboxes}
        \CorrectChoice Propose une façon d'interagir uniformément avec le matériel
        \CorrectChoice Propose des politiques (policy) d'intercommunications entre programmes
        \choice S'exécute comme un processus utilisateur sur le processeur
        \CorrectChoice S'exécute de façon privilégiée sur le processeur
        \choice Sa mémoire est accessible aux programmes utilisateur sans protection
        \CorrectChoice Un \texttt{syscall} fait rentrer en mode kernel
    \end{checkboxes}

    \question[3] Un programme (processus) utilisateur: (0 ou plus choix possibles)
    \begin{checkboxes}
        \CorrectChoice A sa mémoire par défaut isolée des autres programmes
        \CorrectChoice Peut ouvrir des fichiers si le propriétaire du processus a les bons droits
        \choice S'exécute dans le même mode de processeur que le kernel.
        \CorrectChoice S'exécute dans un mode processeur avec moins de droit que le kernel
        \CorrectChoice Si on accède en dehors des zones mémoire allouées (pile ou tas) légitimement on peut obtenir une erreur
        de segmentation
    \end{checkboxes}

    \question[1] Par défaut en C une variable telle que \texttt{int a = 42}\\dans une fonction:
    (0 ou plus choix possibles)
    \begin{checkboxes}
        \choice Est allouée dans le tas (HEAP) avec malloc
        \CorrectChoice Est allouée dans la pile (STACK).
    \end{checkboxes}
  \end{questions}
\end{document}
